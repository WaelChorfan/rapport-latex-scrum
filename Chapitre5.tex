\chapter{ Cl\^{o}ture du projet }

\section{Introduction}

La phase cl\^{o}ture ou de fermeture est la derni\`{e}re phase dans le cycle de d\'{e}veloppement d'un
logiciel avec Scrum. Cette phase est souvent appel\'{e} sprint de stabilisation. Les t\^{a}ches
effectu\'{e}es pendant cette phase pr\'{e}d\'{e}finies, et ils d\'{e}pendent fortement du type de projet.
Pour notre projet, ce chapitre sera consacr\'{e} pour la pr\'{e}sentation des langages et outils de
programmation utilis\'{e}s pour la r\'{e}alisation de nos deux applications.



\section{Impl\'{e}mentation et structure}
Enfin nous d\'{e}crivons les \'{e}tapes globales suivies lors de la r\'{e}alisation de ce projet.
Les \'{e}tapes \'{e}taient :
\bigskip

\begin{itemize}
\item{ Cr\'{e}er le site web statique (front -end) }
\\~~
\item{Cr\'{e}er une application express js }
\\~~
\item{  Cr\'{e}ation du base de donn\'{e}es et liaison des donn\'{e}es par le driver node js
de mysql}
\\~~
\item{Cr\'{e}er un rest api \`{a} l'aide de express js }
\\~~
\item{Int\'{e}gration de front-end avec le back-end }
\\~~
\item{Ajout des modules suppl\'{e}mentaire (Authentification avec jwt et gestion
des roles utilisateur et administrateur) }
\\~~
\item{ H\'{e}bergement en ligne de la base de donn\'{e}es}
\\~~
\item{ H\'{e}bergement en ligne de l'application web}
\\~~
\end{itemize}


\section{Application en ligne}
Vous pouvez trouvez l'application sur le lien [13].


  Pour acc\'{e}der en tant qu'administrateur veuiilez utiliser:

  \begin{itemize}
    \item {\textbf{ Pseudo:}admin}
    \item {\textbf{ Password:}0000}
  \end{itemize}


  Pour acc\'{e}der en tant que membre 'Wael Chorfan' par exemple veuillez utiliser:

  \begin{itemize}
    \item {\textbf{ Pseudo:}WC}
    \item {\textbf{ Password:}0001}
  \end{itemize}

 \newpage



 \section{ Architecture 3 tiers}

Notre projet est caract\'{e}ris\'{e} par son architecture 3 tiers qui inclus un mod\`{e}le MVC (Mod\`{e}le
Vue Contr\^{o}leur)
\textbullet{} Machine: acc\`{e}s et mise \`{a} jour des donn\'{e}es,
\textbullet{} Back-End express Server : contient l'application,
\textbullet{} MySql: gestion de donn\'{e}es et mise \`{a} jour.


\section{Environnement de d\'{e}veloppement}
  \subsection{Environnement mat\'{e}riel }
  Pour la r\'{e}alisation du projet, nous avons utilis\'{e} :
 Un ordinateur portable pour le d\'{e}veloppement ayant les caract\'{e}ristiques suivantes :
\textbullet{} Mod\`{e}le : Asus XJ550.
\textbullet{} Processeur : i7 2.6GHz.
\textbullet{} Disque Dur : 1To
\textbullet{} Syst\`{e}mes d'exploitation : Windows 7.
\textbullet{} M\'{e}moire : 8Go.


  \subsection{Environnement logiciel}

L'environnement logiciel utilis\'{e} pour r\'{e}aliser notre projet est les suivantes :
\textbullet{} MySQL [12]

MySQL est un système de gestion de bases de données relationnelles. Il est distribué
sous une double licence GPL et propriétaire. Il fait partie des logiciels de gestion
de base de données les plus utilisés au monde.


\textbullet{} Visual Studio Code  [1]
est un \'{e}diteur de code extensible d\'{e}velopp\'{e} par Microsoft pour Windows,
Linux et OS X.
Nous avons utilis\'{e} visuel code pour \'{e}crire le code de l'application

\textbullet{}StarUML [14]
StarUML est un logiciel de mod\'{e}lisation UML, c\'{e}d\'{e} comme open source par son \'{e}diteur, \`{a} la
fin de son exploitation commerciale, sous une licence modifi\'{e}e de GNU GPL, Nous avons fait
les diagrammes avec cette technologie.





\section{Outils et technologies }
\textbf{ Node js}
Framework javascript ,nous l'avons utilis\'{e} pour cr\'{e}er le serveur web .
Il offre la rapidit\'{e} de ,la performance etla modularit\'{e}.
\bigskip


\textbf{ Express js:}
framework node js qui sert \`{a} cre\'{e}r l'application ,il est en
relation avec la base de donn\'{e}es par le biais de driver mysql et en relation
avec les modules web par le moteur de vues EJS.
\bigskip
\textbf{ Vue js:}

Framework javascript front-end utilis\'{e} pour la programmation et la
manipulation des actions,entr\'{e}es et sorties des diff\'{e}rents modules .
\bigskip

\textbf{ Mysql:}
Langage de la base de donn\'{e}es relationnelle utilis\'{e}e.
e.EJS moteur de vue d'express js , ce type permet l'intercommunication entre
les modules web et le serveur .
\bigskip

\textbf{ EJS:}
moteur de vue d'express js , ce type permet l'intercommunication entre
les modules web et le serveur .
\bigskip


\textbf{ Highcharts: }
L'outil de \guillemotleft{} reporting \guillemotright{} sur des pages web
\bigskip



\textbf{ Express JS:}

Express js la base de l'application web ,il nous permet de cr\'{e}er l'API REST
qui nous permet de distribuer les elements de l'applications sur des \guillemotleft{} routes \guillemotright{}
ou nous pouvons les acc\'{e}der \`{a} l'aide des middlewares express .
L'authentification est donc faite par un contr\^{o}le sur certains \guillemotleft{} routes \guillemotright{}.
Les middlewarent permettent de communiquer les param\'{e}tres d'entr\'{e}e sortie
entre les pages web d'une part et autres la base de donn\'{e}es d'autre part.























