\chapter{Release 2 :La solution BI }


\section{Introduction}

Dans ce chapitre, il s'agit d'un release qui consiste en un seul sprint qui est la consultation des rapports ,
afin de pouvoir repr\'{e}senter ces rapports ,nous avons besoins d'abord de determiner
les axes de notre \'{e}tude d\'{e}cisonnelle , puis nous allons \'{e}laborer et tester les rapports en utilisant l'outil Power BI
ensuite nous illustrons la partie ETL(Extract Trasform Load) \`{a} l'aide du l'outil Talend,
et nous finissons par la pr\'{e}sentation des rapports obtenues.

\section{ Analyse et conception}
\subsection{Analyse}

\subsubsection{ Diagramme de cas d'utilisation "Consultation des rapports"}
\begin{figure}[H]
\center
\includegraphics[width=12cm,height=5cm]{./figures/ucS.png}
\caption{G\'{e}rer un projet.}

\end{figure}

\subsection{Conception}

\subsubsection{Le sc\'{e}nario \guillemotleft{} Consultation des rapports\guillemotright{}}
Le diagramme de s\'{e}quence \guillemotleft{} Ajout d'une t\^{a}che \guillemotright{} pr\'{e}sente le s\'{e}quencement
des interactions entre Administrateur, Application et Base de donn\'{e}es (BD). (Figure 4.2)


\begin{figure}[H]
\center
\includegraphics[width=14cm,height=9cm]{./figures/seq/G.png}
\caption{Consultation des rapports.}
\end{figure}


\section{ Objectifs de l'\'{e}tude d\'{e}cisionnelle }


\subsection{Axes d'\'{e}tude}
Notre \'{e}tude d\'{e}cisionnelle portera sur :

\begin{itemize}
  \item { Le co\^{u}t total de chaque projet}
  \item { Le co\^{u}t total des projets pour chaque client }
  \item { Le co\^{u}t total par cat\'{e}gorie des projets }
  \item { La dur\'{e}e totale du chaque projet }

\end{itemize}

Pour mettre en evidence cette \'{e}tude  , on recourt \`{a} extraire les rapports bas\'{e}es sur
les Mesures Calcul\'{e}es (Key Performance Indicator),ceux sont des mesures qui permettent aux entreprises d'\'{e}valuer la performance d'un
domaine.( Figure 4.3 ) \newline
Ces mesures int\`{e}grent des tableaux de bord, des outils pour faciliter le suivi et l'aide
\`{a} la d\'{e}cision. Ainsi chaque personne acc\`{e}de aux donn\'{e}es essentielles de gestion pour
piloter son activit\'{e}.



\begin{figure}[H]
\center
\includegraphics[width=11cm,height=12cm]{./figures/kpi.png}
\caption{Key Performance Indicator.}
\end{figure}


\subsection{ Partie ETL }

\subsubsection{Description du Processus ETL }

L'ETL (Extract, Transform, Load) est un processus d'int\'{e}gration des donn\'{e}es qui permet de transf\'{e}rer des donn\'{e}es brutes d'un syst\`{e}me source, de les pr\'{e}parer pour une utilisation en aval et de les envoyer vers une base de donn\'{e}es, un entrep\^{o}t de donn\'{e}es ou un serveur cible. Dans ce processus la transformation des donn\'{e}es intervient sur un serveur interm\'{e}diaire avant le chargement sur la cible comme indiqué
\`{a} la (Figure 4.3 )
\begin{figure}[H]
\center
\includegraphics[width=14cm,height=9cm]{./figures/bi1.png}
\caption{Processus ETL.}
\end{figure}


\subsubsection{D\'{e}finition de data warehouse}
\guillemotleft{}Le Data Warehouse est une collection de donn\'{e}es orient\'{e}es sujet,
int\'{e}gr\'{e}es, non volatiles et \'{e}volutives dans le temps, organis\'{e}es pour le
support d'un processus la d\'{e}cision.\guillemotright{} BILLINMON


\subsubsection{Diff\'{e}rents mod\`{e}les de la mod\'{e}lisation dimensionnelle}
 Mod\`{e}le en \'{e}toile : comme indiqu\'{e} pr\'{e}c\'{e}demment, ce mod\`{e}le se pr\'{e}sente comme
une \'{e}toile dont le centre n'est autre que la table des faits et les branches sont les tables
de dimension. La force de ce type de mod\'{e}lisation est sa lisibilit\'{e} et sa performance.



\begin{itemize}
  \item {Mod\`{e}le en flocon : identique au mod\`{e}le en \'{e}toile, sauf que ses branches sont \'{e}clat\'{e}es en hi\'{e}rarchies. Cette mod\'{e}lisation est g\'{e}n\'{e}ralement justifi\'{e}e par l'\'{e}conomie
d'espace de stockage, cependant elle peut s'av\'{e}rer moins compr\'{e}hensible pour l'utilisateur
final, et tr\`{e}s couteuse en terme de performances.}
  \item {Mod\`{e}le en constellation : Ce n'est rien d'autre que plusieurs mod\`{e}les en \'{e}toile
li\'{e}s entre eux par des dimensions communes. }
\end{itemize}




\subsubsection{Chargement de l'entrepôt  de donn\'{e}es}

Dans notre entrep\^{o}t de donn\'{e}es, nous avons utilis\'{e} le mod\`{e}le en \'{e}toile (Figure 4.5 )

Pour avoir des rapports sur l'application web ,on doit avoir les donn\'{e}es sous une certaine format ,
afin de satisfaire le besoin de la consultation du suivi des projets selon le temps et le co\^{u}t.
Ains les donn\'{e}es correspondantes sont collect\'{e}es, modifi\'{e}es et charg\'{e}es dans un entrep\^{o}t de donn\'{e}es
assurant ainsi la phase ETL.

Nous avons utilis\'{e} l'outil talend pour charger les donn\'{e}es n\'{e}cessaires .
Puisque nous avons les donn\'{e}es ordonn\'{e}es et int\'{e}gres ,nous n'avons pas \'{e}t\'{e} besoin d'une ODS (Operational data source)
nous passont directement \`{a} construite notre table de faits.


\begin{figure}[H]
\center
\includegraphics[width=14cm,height=9cm]{./figures/etoile.png}
\caption{ Mod\`{e}le en \'{e}toile .}
\end{figure}



\subsubsection{Int\'{e}gration}
Pour ce fait ,nous avons utilis\'{e}  l'outil Talend ,les \'{e}l\'{e}ments "tAggregateRow" et l'\'{e}l\'{e}ment "tMap" nous avons permi respectivement
d'extraire les coûts des projets et de rassembler les données provenants des tables "tasks" ,"customers" et "projects")


(Figure 4.6)


Pour obtenir les co\^{u}ts nous avons calcul\'{e} le prix par chaque projet
,cat\'{e}gorie ,et nom du client.(Figure  4.7)

Pour obtenir la dur\'{e}e du chaque projet en jours ,
nous avons calcul\'{e} pour chaque projet la diff\'{e}rence en jours .(Figure 4.8 )

\begin{figure}[H]
\center
\includegraphics[width=14cm,height=10cm]{./figures/integ.png}
\caption{Extraction des co\^{u}ts.1.}
\end{figure}



\begin{figure}[H]
\center
\includegraphics[width=12cm,height=7cm]{./figures/integ1.png}
\caption{Extraction des co\^{u}ts.2.}
\end{figure}







\begin{figure}[H]
\center
\includegraphics[width=12cm,height=7cm]{./figures/integ2.png}
\caption{Extraction des  dur\'{e}es.}
\end{figure}



\newpage

\subsubsection{Sh\'{e}mas obtenus}
A partir du du sh\'{e}ma de la Figure 4.9 nous avons obtenue la table fact selon le sh\'{e}ma dans la Table 4.1,
et les donn\'{e}es sont alors dans la table 4.2

\FloatBarrier

\begin{figure}[H]
\center
\includegraphics[width=14cm,height=9cm]{./figures/s1.png}
\caption{Sh\'{e}ma initial.}
\end{figure}



\begin{table}

\begin{tabular}{|l|l|l|l|l|l|l|}
\hline
 \textbf{Field}        & \textbf{Type}         & \textbf{Null} & \textbf{Key}  & \textbf{Default} & \textbf{Extra}   \\
\hline
price        & double(22,0) & YES  &     & NULL    &         \\
\hline
projectname  & varchar(45)  & NO   &     & NULL    &         \\
\hline
duration     & bigint(10)   & NO   &     & NULL    &         \\
\hline
customername & varchar(45)  & YES  &     & NULL    &         \\
\hline
category     & varchar(60)  & YES  &     & NULL    &         \\
\hline
\end{tabular}
\centering
\caption{Sh\'{e}ma de la table des faits.}
\end{table}



les donn\'{e}es obtenues sont comme suit :

\begin{table}

\begin{tabular}{|l|l|l|l|l|}
\hline
price & projectname                   & duration & customername            & category  \\
\hline
66    & Website wordpress             & 63       & ITBS Nabeul             & Hardware  \\
\hline
121   & Application mobile            & 31       & Proin Velit Corporation & Software  \\
\hline
81    & Video Making                  & 81       & ITBS Nabeul             & Software  \\
\hline
43    & Create geolocalisation system & 27       & Lobortis Quis Pede LLP  & Hardware  \\
\hline
\end{tabular}
\centering
\caption{Contenu de la table des faits.}
\end{table}


\section{Test}


Si l'administrateur parvient \`{a} ese connecter il trouve cette interface d'accueil
ou il trouvera les rapports qui d\'{e}crivent des statistiques primordiales au
d\'{e}oulement des projets donc les interfaces et les fonctionnalit\'{e}s disponibles.

\subsection{  L'analyse des donn\'{e}es avec   l'outil Power BI }


\subsubsection{Pr\'{e}paration}
Nous avons besoin d'ajuster quelques champs pour \^{e}tre ad\'{e}quates \`{a} \^{e}tre
exploit\'{e}es . Par exemple dans la table membre on trouve " firstname " et
"lastname" Et on doit les concat\'{e}ner pour dans une hi\'{e}rarchie pour avoir
des rapports lisibles \`{a} propos des membres .





\begin{figure}[H]
\center
\includegraphics[width=14cm,height=5cm]{./figures/pb1.png}
\caption{Ajustement des noms.}
\end{figure}



\bigskip
\bigskip


\begin{figure}[H]
\center
\includegraphics[width=14cm,height=3cm]{./figures/pb2.png}
\caption{Ajustement de la dur\'{e}e totale des projets.}
\end{figure}


Pour tester les rapports et la validit\'{e} des donn\'{e}es , nous avons utilis\'{e} l'outil
Power BI pour nous donner une impression sur les rapports qu'on doit
int\'{e}grer \`{a} notre application afin de les visulaiser.

\bigskip
Pour ce fait on a cr\'{e}e les rapports correspontdants \`{a} l'\'{e}tude d\'{e}cisionnelle :
Les deux axes importants de l'\'{e}tude sont les couts et la dur\'{e}e :


\newpage
\subsubsection{R\'{e}sultats  }
\begin{itemize}

\item{ \textbf{Les  co\^{u}ts}  }

\begin{figure}[H]
\center
\includegraphics[width=8cm,height=5cm]{./figures/rpb1.png}
\caption{Co\^{u}t total par projet.}

\end{figure}

\begin{figure}[H]
\center
\includegraphics[width=6cm,height=5cm]{./figures/rpb2.png}
\caption{Co\^{u}t total par catégorie de projet.}

\end{figure}

\begin{figure}[H]
\center
\includegraphics[width=8cm,height=5cm]{./figures/rpb3.png}
\caption{Co\^{u}t total par client.}

\end{figure}

\begin{figure}[H]
\center
\includegraphics[width=8cm,height=5cm]{./figures/rpb31.png}
\caption{Co\^{u}t total par client en chiffres.}

\end{figure}




\newpage

\item{ \textbf{La dur\'{e}e}  }

\begin{figure}[H]
\center
\includegraphics[width=8cm,height=5cm]{./figures/rpb4.png}
\caption{Dur\'{e}e total par projet (en jours).1}

\end{figure}

\begin{figure}[H]
\center
\includegraphics[width=8cm,height=5cm]{./figures/rpb42.png}
\caption{Dur\'{e}e total par projet (en jours).2}

\end{figure}



\end{itemize}



\subsection{La consultation des rapports  sur l'application web}

\begin{figure}[H]
\center
\includegraphics[width=11cm,height=7cm]{./figures/pres/cost-per-project.png}
\caption{ Co\^{u}t par projet }
\end{figure}



\begin{figure}[H]
\center
\includegraphics[width=11cm,height=7cm]{./figures/pres/cost-per-customer.png}
\caption{ Co\^{u}t par client }
\end{figure}

\begin{figure}[H]
\center
\includegraphics[width=11cm,height=7cm]{./figures/pres/cost-per-category.png}
\caption{ Co\^{u}t par cat\'{e}gorie}
\end{figure}



\begin{figure}[H]
\center
\includegraphics[width=11cm,height=7cm]{./figures/pres/duration-per-project.png}
\caption{Dur\'{e}e par projet.1. }
\end{figure}



\begin{figure}[H]
\center
\includegraphics[width=11cm,height=7cm]{./figures/pres/durationin-days-per-proj.png}
\caption{Dur\'{e}e par projet.2.}
\end{figure}



\section{Conclusion}

A ce stade, nous avons réussi  à la réalisation de la solution BI ,l'administrateur peut maintenant
consulter les rapports issues de l'étude décisionnelle achevée lors de ce sprint. 