

\chapter*{Abbr\'{e}viations}
\bigskip

\textbf{UML: } 
Unified Modeling Language
\newline

\textbf{JS: } 
JavaScript
\newline

\textbf{BD: } 
Base de données
\newline

\textbf{ BI:} 
Business Intelligence
\newline

\textbf{ORM: } 
Object Relational Mapping
\newline

\textbf{JWT: } 
Json Web Tokens
\newline

\textbf{EJS: } 
Embedded JavaScript
\newline




%\addcontentsline{toc}{chapter}{Introduction g\'{e}n\'{e}rale}
\chapter*{Introduction g\'{e}n\'{e}rale}


Il ne fait d\'{e}sormais plus aucun doute que l'informatique est la r\'{e}volution la
plus importante et la plus innovante qui a marqu\'{e} la vie de l'humanit\'{e}
moderne. En effet, les logiciels informatiques proposent maintenant des
solutions \`{a} tous les probl\`{e}mes de la vie, aussi bien dans des domaines
professionnels que pour des applications personnelles. Et leurs m\'{e}thodes de
conception et de d\'{e}veloppement ont vu l'av\`{e}nement d'autant de technologies
qui facilitent leur mise en place et leurs donnent des possibilit\'{e}s et des
fonctionnalit\'{e}s de plus en plus \'{e}tendues.


\bigskip
Une start-up repr\'{e}sente une des organisations ayant des ressources et des
activit\'{e}s dont la gestion n\'{e}cessite une application informatique. Ainsi,
l'objectif de notre projet est de r\'{e}aliser une application informatique
interactive, fiable, conviviale et facile \`{a} int\'{e}grer dans l'environnement de
travail, assurant la gestion des syst\`{e}mes d'informations compte tenu des
besoins exprim\'{e}s. Cette application vise essentiellement \`{a} diminuer la
complexit\'{e} des traitements ainsi que le temps perdu lors de la gestion des
projets, en particulier.

\bigskip


Dans le pr\'{e}sent rapport, nous pr\'{e}sentons en d\'{e}tail les \'{e}tapes que nous avons
suivies pour r\'{e}aliser notre application. Ce rapport comporte quatre chapitres
qui sont organis\'{e}s comme suit:

\bigskip
\textbf{Chapitre 1:}

Intitul\'{e} \'{e}tude pr\'{e}alable qui consiste \`{a} souligner le contexte de projet,
pr\'{e}senter l'organisme d'accueil ainsi que la th\'{e}matique de projet dans
laquelle on va citer la probl\'{e}matique, les solutions existantes et la solution
propos\'{e}e et on finit par d\'{e}crire l'architecture que nous avons adopt\'{e} pour
l'application en illustrant nos choix techniques

\bigskip
\textbf{Chapitre 2:}

Pr\'{e}sente une \'{e}tape primordiale, les sp\'{e}cifications des besoins, la mod\'{e}lisation
et l'\'{e}tude conceptuelle. C'est \`{a} ce niveau que nous avons \'{e}voqu\'{e} l'aspect
conceptuel de notre application en commen\c{c}ant par sp\'{e}cifier les besoins et les
acteurs de syst\`{e}mes et on passant par les diagrammes UML qui sont le
diagramme de cas d'utilisation, le diagramme de s\'{e}quence, le diagramme
d'activit\'{e} et le diagramme de d\'{e}ploiement.

\bigskip
\textbf{Chapitre 3:}

Dans ce chapitre, on va d\'{e}tailler les entit\'{e}s qui composent notre projet ce qui
implique l'illustration des entit\'{e}s de la base de donn\'{e}es ce qui va nous
ramener \`{a} d\'{e}finir notre diagramme entit\'{e}s-relations.
Et en fin apr\`{e}s construire notre base de donn\'{e}es, nous allons entamer la partie
BI dont on va mod\'{e}liser notre base de donn\'{e}es afin d'\^{e}tre capable de
l'exploiter et de la transformer en des rapports significatifs.

\bigskip
\textbf{Chapitre 4:}

Pour finir nous encha\^{\i}nons avec le chapitre de R\'{e}alisation qui est consacr\'{e} \`{a} la
pr\'{e}sentation de l'environnement mat\'{e}riel et logiciel utilis\'{e} pour la r\'{e}alisation
de notre application, en premier lieu. En second lieu, nous avons pr\'{e}sent\'{e} les
choix techniques adopt\'{e}s ainsi que la solution propos\'{e}e tout en s'aidant des
interfaces graphiques, qui comportent une illustration graphique de
l'application de point de vue du profil d'un utilisateur donn\'{e}, avec bien s\^{u}r une
description des choix ergonomiques adopt\'{e}s






