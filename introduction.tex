






%\addcontentsline{toc}{chapter}{Introduction g\'{e}n\'{e}rale}
\chapter*{Introduction g\'{e}n\'{e}rale}


Il ne fait d\'{e}sormais plus aucun doute que l'informatique est la r\'{e}volution la
plus importante et la plus innovante qui a marqu\'{e} la vie de l'humanit\'{e}
moderne. En effet, les logiciels informatiques proposent maintenant des
solutions \`{a} tous les probl\`{e}mes de la vie, aussi bien dans des domaines
professionnels que pour des applications personnelles. Et leurs m\'{e}thodes de
conception et de d\'{e}veloppement ont vu l'av\`{e}nement d'autant de technologies
qui facilitent leur mise en place et leurs donnent des possibilit\'{e}s et des
fonctionnalit\'{e}s de plus en plus \'{e}tendues.


\bigskip
Une start-up repr\'{e}sente une des organisations ayant des ressources et des
activit\'{e}s dont la gestion n\'{e}cessite une application informatique. Ainsi,
l'objectif de notre projet est de r\'{e}aliser une application informatique
interactive, fiable, conviviale et facile \`{a} int\'{e}grer dans l'environnement de
travail, assurant la gestion des syst\`{e}mes d'informations compte tenu des
besoins exprim\'{e}s. Cette application vise essentiellement \`{a} diminuer la
complexit\'{e} des traitements ainsi que le temps perdu lors de la gestion des
projets, en particulier.

\bigskip


Dans le pr\'{e}sent rapport, nous pr\'{e}sentons en d\'{e}tail les \'{e}tapes que nous avons
suivies pour r\'{e}aliser notre application. Ce rapport comporte quatre chapitres
qui sont organis\'{e}s comme suit:

\bigskip
\textbf{Chapitre 1:}

Intitul\'{e} \'{e}tude pr\'{e}alable qui consiste \`{a} souligner le contexte de projet,
pr\'{e}senter l'organisme d'accueil ainsi que la th\'{e}matique de projet dans
laquelle on va citer la probl\'{e}matique, les solutions existantes et la solution
propos\'{e}e en passant par la description de l'architecture que nous avons adopt\'{e} pour
l'application en illustrant nos choix techniques. Ensuite nous finissons
par le choix de la méthodologie du travail(SCRUM)  et le planning du projet.

\bigskip
\textbf{Chapitre 2:}

Pr\'{e}sente une \'{e}tape primordiale, les sp\'{e}cifications des besoins, la mod\'{e}lisation
et l'\'{e}tude conceptuelle. C'est \`{a} ce niveau que nous avons \'{e}voqu\'{e} l'aspect
conceptuel de notre application en commen\c{c}ant par sp\'{e}cifier les besoins et les
acteurs de syst\`{e}mes et en passant par la pr\'{e}cision des \'{e}l\'{e}ments du "Product Backlog"

\bigskip

\textbf{Chapitre 3:}
Ce chapitre est consacr\'{e} à la r\'{e}alisation de release 1 ,qui est principalement la r\'{e}alisation de l'application
web .On va d\'{e}tailler l'analyse qui consiste à pr\'{e}senter le
diagramme de cas d'utilisation, ,la conception
qui consiste à pr\'{e}senter le diagramme de s\'{e}quence,
et nous passons par les tests correspondantes de chaque sprint.

Ainsi pour une meilleur explication de l'application web, nous pr\'{e}sentons les diagrammes
d'activit\'{e} et de d\'{e}ploiement.
\newline
\bigskip
\textbf{Chapitre 4:}

Ce chapitre est consacr\'{e} à la r\'{e}alisation de release 2 ,qui est principalement la r\'{e}alisation de la partie BI (Business Intelligence).
On va d\'{e}tailler l'analyse ,la conception et nous passons  par les tests correspondantes de sprint tout en s'aidant des
interfaces graphiques.

Lors de l'\'{e}tude décisonnelle nous avons mod\'{e}lis\'{e} notre base de donn\'{e}es afin d'\^{e}tre capable de
l'exploiter et de la transformer en des rapports significatifs.


\textbf{Chapitre 5:}
Pour finir nous encha\^{\i}nons avec le chapitre de cl\^{o}ture qui est consacr\'{e} \`{a} la
pr\'{e}sentation de l'environnement mat\'{e}riel et logiciel utilis\'{e} pour la r\'{e}alisation
de notre application, en premier lieu. En second lieu, nous  pr\'{e}sentons les
choix techniques adopt\'{e}s  avec bien s\^{u}r une
description des choix ergonomiques adopt\'{e}s.






