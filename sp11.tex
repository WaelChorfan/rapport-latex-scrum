 Selon la planification \'{e}tablie, le premier Sprint porte principalement sur le cas
d'utilisation \guillemotleft{} G\'{e}rer un planning d'affectation\guillemotright{} que nous allons d\'{e}cortiquer par la suite.
L'objectif de cette premi\`{e}re it\'{e}ration est de fournir une interface qui va permettre \`{a}
l'administrateur d'ajouter, mettre \`{a} jour et consulter les informations et les d\'{e}tails des membres.

\subsection{Analyse}

 Cas d'utilisation \guillemotleft{} G\'{e}rer un planning d'affectation\guillemotright{}
Dans cette activit\'{e} nous allons raffiner les cas d'utilisation prioritaire et les d\'{e}crire en d\'{e}tail
afin de mieux visualiser notre application.
D\'{e}tailler le cas d'utilisation \guillemotleft{}G\'{e}rer un planning d'affectation\guillemotright{} revient \`{a} d\'{e}tailler 
ses sous cas d'utilisation \`{a} savoir :
\textbullet{} Ajouter un planning d'affectation,
\textbullet{} Mettre \`{a} jour un planning d'affectation


\subsection{Conception}

 Diagramme de séquence gestion membres

\subsection{Codage}

 Dans cette activité nous passons au développement et à l’implémentation des cas d’utilisation
analysés et conçus.
Nous nous intéresserons alors de plus près au schéma de la base de données en présentant la
table « Planification» tout en tenant compte du type de leur attribut.

Table membres

\subsection{Test}
Contrairement aux cycles de développement séquentiel, avec la méthodologie agile, le test n'est
pas une phase qui se déroule après la fin de développement. En effet, les tests seront intégrés
dès le début du premier sprint jusqu’à la livraison du produit final.
Dans ce premier sprint, nous avons testé la fonctionnalité « Gérer Receveur» qui affiche, ajout,
supprime les receveurs de la base de données.
Ci-dessus nous ajoutons des captures écran du sprint réalisé.