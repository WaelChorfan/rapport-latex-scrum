
\chapter*{Conclusion et Perspectives}
\addcontentsline{toc}{chapter}{Conclusion et Perspectives}
Notre projet  consiste \`{a} la conception et la r\'{e}alisation d'une plateforme web destin\'{e} pour la de gestion des projets
Contrairement \`{a} la majorit\'{e} des travaux existants sur le march\'{e} qui n\'{e}cessitent un effort de configuration consid\'{e}rable, nous avons r\'{e}alis\'{e} un syst\`{e}me qui permet \`{a} la fois de g\'{e}rer les projets et g\'{e}n\'{e}rer des rapports pour avoir une id\'{e}e g\'{e}n\'{e}rale sur le d\'{e}roulement des projets en cours.

En ce qui concerne la d\'{e}marche, nous avons en premier lieu effectu\'{e} une phase d'\'{e}tude des diff\'{e}rents outils existants. En deuxi\`{e}me lieu nous avons sp\'{e}cifi\'{e} notre application pour discerner les fonctionnalit\'{e}s .En troisi\`{e}me lieu, nous avons proc\'{e}d\'{e} \`{a} sa conception ainsi qu'aux choix technologiques pour sa r\'{e}alisation. Enfin, nous l'avons mise en \oe{}uvre.
Toutes les fonctionnalit\'{e}s d\'{e}crites dans le cahier des sp\'{e}cifications fonctionnelles ont \'{e}t\'{e} d\'{e}velopp\'{e}es et valid\'{e}es. N\'{e}anmoins, notre projet pourra \^{e}tre am\'{e}lior\'{e} par l'ajout d'autre
fonctionnalit\'{e}s comme :

\begin{itemize}
\item{ La gestion par heure et non par date seulement.}
\item{ L'ajout des comptes d'utilisateurs pour les clients afin qu'il suivent l'avancement de leurs projets.}
\item{ La suivie des erreurs et des probl\'{e}mes qui sera une  cons\'{e}quence de la gestion des tickets lanc\'{e}es par les clients.  }
\end{itemize}


Il est important \`{a} noter que la r\'{e}alisation de ce projet noua a \'{e}t\'{e} b\'{e}n\'{e}fique sur tous les plans.
Sur le plan technique, ce projet nous a \'{e}t\'{e} une bonne occasion pour d\'{e}couvrir et maitriser la technologie node js et vue js ,
d'approfondir nos connaissances sur le plan des nouvelles technologies de d\'{e}veloppement web et
d' h\'{e}bergement des applications en ligne avec un compte gratuit limit\'{e} sur la plateforme \guillemotleft{} heroku \guillemotright{}.

Sur le plan humain, ce projet a \'{e}t\'{e} une v\'{e}ritable occasion de vivre de pr\'{e}s l'exp\'{e}rience du travail au sein d'une soci\'{e}t\'{e}, qui exige la ponctualit\'{e} et l'int\'{e}gration dans un groupe de travail.

Ce qui nous a permis d'am\'{e}liorer nos capacit\'{e}s de communication et de nous adapter \`{a} la vie professionnelle. Nous avons fait de notre mieux pour bien laisser une bonne impression sur notre discipline, nos qualit\'{e}s et nos comp\'{e}tences techniques vis \`{a} vis du staff technique de start-up Cherchini.tn et pr\'{e}senter un travail \`{a} la hauteur de la formation qui nous avons
eu au sein de l'ITBS.

\chapter*{Abbr\'{e}viations}
\addcontentsline{toc}{chapter}{Abbr\'{e}viations }
\bigskip

\textbf{UML: }
Unified Modeling Language
\newline

\textbf{JS: }
JavaScript
\newline

\textbf{BD: }
Base de données
\newline

\textbf{ BI:}
Business Intelligence
\newline

\textbf{ORM: }
Object Relational Mapping
\newline

\textbf{JWT: }
Json Web Tokens
\newline

\textbf{EJS: }
Embedded JavaScript
\newline


\textbf{HTML : }
HyperText Markup Language
\newline


\chapter*{Webographie}
\addcontentsline{toc}{chapter}{Webographie }

\begin{itemize}
  \item {[1]  \url{https://code.visualstudio.com/download }  }
  \item { [2]  \url{https://nodejs.org/en/ }  }
  \item {  [3]  \url{https://www.npmjs.com} }
  \item {  [4]  \url{http://www.wampserver.com}}
  \item {  [5]  \url{https://www.phpmyadmin.net}  }
  \item {  [6]  \url{https://remotemysql.com}  }
  \item {  [7]  \url{https://www.heroku.com}  }
  \item { [8]  \url{https://getbootstrap.com/docs/4.3/getting-started/introduction/}   }
  \item { [9]  \url{https://stackoverflow.com/} }
  \item {  [10] \url{https://ejs.co/#docs} }
  \item {  [11] \url{https://vuejs.org/v2/guide/} }
  \item {  [12] \url{https://dev.mysql.com/doc/refman/5.7/en/}  }
  \item {   [13] \url{https://cherchini-project.herokuapp.com} }
  \item {   [14] \url{http://staruml.io/}  }
  \item {   [15] \url{https://www.highcharts.com}  }
  \item {  [16] \url{https://www.getpostman.com/} }
  \item {  [17] \url{https://fr.talend.com/} }
 \end{itemize} 